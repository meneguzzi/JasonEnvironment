\documentclass[a4paper]{report}
\usepackage{hyperref}


\title{Jason Environments Library}
\author{Felipe Meneguzzi}

\begin{document} 
\maketitle

\chapter{Introduction}

Jason is Java-based interpreter of an extended version of the AgentSpeak(L)
agent language. 
This software package allows the development multiagent systems
through support of various communication libraries such as JADE and SACI. 
Additions to the basic interpreter are created by extending the Java classes
that comprise the interpreter. 
Additional internal actions cab be create by extending the
\texttt{InternalAction} class, while simulations are created by extending the
\texttt{Environment} class.

One of the main tasks of a programmer when creating an agent simulation is the
creation of the environment. 
In order to ease the burden of the programmer when the intention is to simply
create a simulation using first-order logic (as opposed to creating complex,
UI-interacting environments), we have created a library of classes to implement
a \emph{modular} environment.

Main classes:
\begin{itemize}
  \item \texttt{ModularEnvironment} ---  A modular environment class for Jason,
  allowing easy development of new environments for Jason, including the concept
  of an \emph{external} action, similar in development concept to Jason's
  internal actions, but this one used in the environment, thus avoiding
  concentrating the action implementation on the \texttt{executeAction(String,
  Structure)} method from \texttt{Environment}.
  \item \texttt{ScriptedEnvironment} --- An environment class that allows
  \emph{scripts} of events to be supplied from external files, allowing one to
  repeat simulations exactly. This class is useful for repeated experiments of
  testing of a single agent's behaviour.
  \item \texttt{ExternalAction} --- An interface for an action executable from
  within a certain modular environment.
  \item \texttt{StripsAction} --- An external action that represents a STRIPS
  operator (and can be parsed from a STRIPS specification).
\end{itemize}

\chapter{Configuration}

Multiagent systems in Jason are created through a project configuration file in
the MAS2J format \cite{Bordini2007}. 

\begin{verbatim}
MAS <ID> {
    [infrastructure]
    [environment]
    [exec_control]
    agents
} 
\end{verbatim}

\chapter{Usage}

\bibliography{jasonenv}
\bibliographystyle{plain}

\end{document}